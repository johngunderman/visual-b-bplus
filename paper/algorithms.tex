\section{Algorithms}

\subsection{B and B+ Algorithms}
% kenny, put a section here about B+ algorithms

\subsection{Graph Layout Algorithms}
The UI portion of this project requires a sizable amount of
algorithmic work. The most difficult factor is the development of a
node layout algorithm that prevents node overlap, while also giving an
ideal spacing between the nodes in the tree. This is especially
important as the size of the tree grows.

A popular method for solving this problem is the use of the
force-based algorithm. This algorithm works well for spreading nodes
out over 2 or 3 dimensional space. However, it does not take into
account the movement limitations that are imposed in a B+
tree. Namely, the nodes in a B+ tree can only move horizontally except
for in the case of a node split or join.

A secondary issue with the use of force-based algorithms is that they
have a running time of O$(n^3)$. This long running time is not ideal
for large trees. Our design helps mitigate the issues presented by
this running time. Namely, the size of the tree that can be visible to
the user is limited by the available screen real estate. This limits
the number of nodes down to a reasonable level, and thus keeps our
running time within reason.

Force based algorithms operate by modelling interaction between the
graph nodes using equations borrowed from physics. Each node is
treated as an electrically charged particle, repelling the other nodes
in the graph according to Coulomb's Law. Each edge on the graph is
treated as a spring, attracting the two connected nodes according to
Hooke's Law. The simulation randomizes initial node positions and then
simulates each timestep in the model until equilibrium is reached. At
equilibrium, the nodes should be placed in such a way that overlap is
minimized.

%% what we did
Adapting this algorithm to work with our constraints proved to be an
interesting challenge. Of primary concern was the limitation that each
node should stay on its own level in the tree throughout the
simulation. This problem turned to be simple to solve. We modified the
initial force-based algorithm so that nodes could only apply force on
the X coordinate plane.

After implementing this system, it became clear that there were still
some serious issues to address. The nodes in the tree were becoming
far to spread apart. Furthermore, they were also off-center from their
expected locations. After some experimentation, it was determined that
the issue sprouted from an assumption the original algorithm made. The
original algorithm assumed that all nodes would repel each-other, as
would make sense in a 2-dimensional model. Unfortunately, the same
rule did not apply for 1-dimensional models. The solution to this
problem was to modify the code in such a way that the nodes in the
model would only repel against other nodes on that level of the tree.

%% WIP
