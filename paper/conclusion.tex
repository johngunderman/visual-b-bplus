\section{Future Work}

Future changes we would like to make to our project would include a
hefty redesign for some of the tre structure algorithms, specifically
the deletion methods.  Our current deletion algorithms do a complete
restructuring of the tree when we encounter an underflow because of
deletion.  This method is convenient for reducing lines of code, but
can become very costly for large trees.  As such in the future we hope
to be able to implement the more common method for handling deletion
that results in underflow, which is to then cascade upward, merge with
or borrowing from sibling nodes.  In the general case this method
would prove to have better performance, having a deletion time of
O(log n) as oppposed to our current method which has a deletion time
of O(n log n).  Part of the reason we didn't not do the standard
method of deletion initially was because of some of the limitation fo
Javascript.  Javascript lacks easy to use pointers, which made things
a bit harder when trying to keep track for each node/block, what it's
children and parent were.  As such we could not find a reasonable
method of linking sibling nodes which would be necessary for efficient
implementation of standard method of handling underflow by borrowing
or merging with siblings.  We'd also like to better organize the
modularization of our insertion methods, while they work, and run in
time O(log n) the code itself is a bit disorganized making it hard to
read through if you were trying to go through it manually to verify
correctness.

Visually, it would also be a nice addition if we could provide step by
step animation of how values, or a search is moving through the tree
to provide a better means of understanding exactly what is occurring
during various search, insertion, or deletion queries.  A simple
textual output could suffice if it explicitly stated things like:
"Key 3 removed from node X"
"Node X borrowed values a,b from node Y to cover underflow"
...
"Key 6 inserted into Node X"
"Node X split into Nodes Y, with values a,b, and Z, with values c,d"
etc.

This atleast would allow the user to follow along with the action step
by step, even if they could not see if visually animated.

\section{Conclusion}

Our project successfully met the goals we set out to accomplish when
we began this project. The user can successfully insert, delete, and
search for values in the B or B+ Tree. At each action, the user
recieves a visualization of the new tree state.

In the process of develooping this project, we came across more great
ideas for features we could develop. These are detailed above, in the
\textit{Future Work} section.
